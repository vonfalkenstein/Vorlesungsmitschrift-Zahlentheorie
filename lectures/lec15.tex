\begin{rem*}\lecture{01.06.2021}
	Warum benötigen wir $+\epsilon$ in der Vermutung?\\
	Hierfür betrachten wir $a=1,b=3^{2^n}-1,c=3^{2^n}$. Dann ist $a+b=c$ mit $\ggt(a,b,c)=1$. Es gilt
	\[ 2^{n+1} \mid 3^{2^n}-1. \]
	Also
	\begin{align*}
		\rad(abc) &= \rad \left( \left( 3^{2^n}-1 \right) 3^n \right)\\
		&= \rad \left( 3^{2^n} -1 \right) \rad(3^n)\\
		&\leq \frac{3^{2^n}-1}{2^n} \cdot 3\\
		&\leq \frac{3c}{2^n}
	\end{align*}
	Es folgt also
	\[ c > \frac{2^n}{3} \rad(abc). \]
\end{rem*}

\subsubsection*{Was ist bekannt zur $abc$-Vermutung?}

\begin{thm*}[Stewart \& Kunrui Yu, 1991]
	Für jedes \( \epsilon>0 \) gibt es eine Konstante $C_0(\epsilon)$, sodass gilt: Sind $a,b,c \in \N$ mit $a+b=c$ und $\ggt(a,b,c)=1$, so ist
	\[ c < \max \left\{ C_0(\epsilon), e^{\rad(abc)^{\frac{1}{3}+\epsilon}} \right\}. \]
\end{thm*}

\begin{thm*}[Pastén, 2017]
	Für jedes $\epsilon>0$ gibt es eine Konstante $C_0(\epsilon)$ mit folgender Eigenschaft: Sind $a,b,c \in \N$ mit $a+b=c$ und $\ggt(a,b,c)=1$, so gilt
	\[ d(abc) < C_0(\epsilon) \cdot \rad(abc)^{\frac{8}{3}+\epsilon}, \]
	wobei $d$ die Teilerfunktion ist.
\end{thm*}

\begin{conj*}[Schwache $abc$-Vermutung]
	Es gibt eine Konstante $\gamma>0$, sodass für alle $a,b,c \in \N$ mit $a+b=c$ und $\ggt(a,b,c)=1$ gilt
	\[ c \leq \rad(abc)^\gamma. \]
\end{conj*}

2012 behauptete Mochizuki\footnote{Shin’ichi Mochizuki (geb. 1969), ein japanischer Mathematiker}, die $abc$-Vermutung könne mit einer von ihm entwickelten "interuniversellen Teichmüller\footnote{nach Oswald Teichmüller (1913-1943), ein dummer Nazi}-Theorie". Dies erregte viel Aufmerksamkeit, allerdings widerlegten 2018 Scholze\footnote{Peter Scholze (geb. 1987), ein deutscher Mathematiker} und Stix\footnote{Jakob Stix (geb. 1974), ein deutscher Mathematiker} die Beweisideen von Mochizuki.

\section{Folgerungen aus der $abc$-Vermutung}

%\subsection{Quadratfreie Werte von Polynomen}

\begin{defn*}[Quadratfreie Zahl]\index{Quadratfreie Zahl}
	Wir nennen eine natürliche Zahl $m \in \N$ \emph{quadratfrei}, falls für jede Primzahl $p$ mit $p \mid m$ zusätzlich $p^2 \nmid m$ gilt.
\end{defn*}

\begin{frage*}
	Für welche natürlichen Zahlen $n \in \N$ ist $ n^4+1 $ quadratfrei?
\end{frage*}

\begin{conj*}
	\[ \left| \left\{ 1 \leq n \leq X \mid n^4+1 \ \text{ist quadratfrei} \right\} \right| \cong X \prod_p \left( 1-\frac{C_p}{p} \right) \]
	mit 
	\[ C_p = \left| \left\{ n \bmod p^2 \mid p^2 \mid n^4+1 \right\} \right| \]
\end{conj*}

\begin{thm*}[Granville, 1998]
	Angenommen die $abc$-Vermutung gilt. Sei $f \in \Z[x]$. Dann nimmt $f$ die erwartete Zahl von quadratfreien Werten an.
\end{thm*}

\begin{thm*}[Poonen\protect\footnotemark{}, 2003]
	\footnotetext{Bjorn Poonen (geb. 1968), ein US-amerikanischer Mathematiker}
	Granvilles Satz gilt auch für Polynome in mehreren Variablen.
\end{thm*}

\begin{frage*}
	Bestimmt alle $m,n \in \N$ mit $n!+1=m^2$.
\end{frage*}

Lösungen, die wir kennen, sind beispielsweise \( 4!+1=5^2, 5!+^=11^2, 7!+1=71^2,\dots \)\\
Angenommen, $m,n \in \N$ mit $n!+1=m^2$. Dann gilt
\begin{align*}
	n! &= m^2-1 \\ &=(m-1)(m+1)
\end{align*}
Ist $n \geq 2$, dann sind $m-1$ und $m+1$ beide gerade und aus Primfaktoren $\geq n$ aufgebaut. Wir wissen
\begin{align*}
	\frac{m-1}{2}+1 = \frac{m+1}{2}
\end{align*}
Berechne
\[ \rad \left( \frac{m-1}{2} \cdot \frac{m+1}{2} \right) \leq \prod_{p \leq n} p.\]
Wenn die schwache $abc$-Vermutung mit einem $\gamma>0$ gilt, so erhalten wir
\begin{align*}
	\frac{m+1}{2} &< \left( \rad \left( \frac{m-1}{2} \frac{m+1}{2} \right) \right)^\gamma\\
	&\leq \left( \prod_{p \leq n} p \right)^\gamma
\end{align*}
Man kann zeigen
\[ \prod_{p \leq n}p < 4^n. \]
Daraus folgt
\[ \frac{\sqrt{n!}}{2} < \frac{m+1}{2} < 4^{\gamma n}. \]
Gilt die schwache $abc$-Vermutung, dann gibt es höchstens endlich viele Lösungen der Gleichung
\[ n!+1 = m^2. \]

\begin{conj*}[Schwache Hall-Vermutung]
	Sei $0 < \delta < \frac{1}{2}$. Dann gibt es höchstens endlich viele Paare $x,y \in \N$ mit
	\[ 0 < |x^3-y^2| < x^\delta. \]
\end{conj*}

\begin{thm*}
	Die $abc$-Vermutung impliziert die schwache Hall-Vermutung.
\end{thm*}

\begin{frage*}
	Gibt es unendlich viele Primzahlen $p$, sodass $2{p-1} \equiv 1 \bmod p^2$?
\end{frage*}

\begin{rem*}
	Ist $p$ ungerade, so gilt nach Eulers Satz
	\[ 2^{p-1} \equiv 1 \bmod p. \]
\end{rem*}

\begin{thm*}[Silverman\protect\footnotemark{}, 1988]
	\footnotetext{Joseph Hillel Silverman (geb. 1955), ein US-amerikanischer Mathematiker}
	Angenommen die $abc$-Vermutung gilt. Dann gibt es unendlich viele Primzahlen $p$ mit $2^{p-1} \not\equiv 1 \bmod p^2$.
\end{thm*}