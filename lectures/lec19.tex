\section{Gauß-Summen}\lecture{15.06.2021}

Sei $m \in \N$. Bisher haben wir Charaktere
\[ \tilde{\chi}: (\Z/m\Z)^* \to \C^* \]
studiert, die multiplikativen Charaktere modulo $m$. Wir betrachten nun die additive Gruppe $\Z/m\Z$ und  Gruppenhomomorphismen
\[ (\Z/m\Z,+) \to (\C^*,\cdot). \]

\begin{frage*}
	Welche Form haben "additive Charaktere" modulo $m$?
\end{frage*}

Sei 
\[ f: (\Z/m\Z,+) \to (\C^*,\cdot) \]
ein Gruppenhomomorphismus. Dann gilt
\[ f(0) = 1 \]
\[ f(1)^m = f(m) = f(0) = 1, \]
d.h. $f(1)$ ist eine $m$-te Einheitswurzel. Es gibt also $a \in \Z$ mit
\[ f(1) = e^{2\pi i \frac{a}{m}}. \]
Dann gilt allgemein für $c \in \Z/m\Z$
\[ f(c) = f(1)^c = e^{2\pi i \frac{ac}{m}} \]

\begin{notat*}
	Für $x \in \R$ schreiben wir im Folgenden
	\[ e(x) = e^{2\pi i x}. \]
\end{notat*}

\begin{frage*}
	Wie sehen die Interaktionen zwischen additiven und multiplikativen Charakteren modulo $m$ aus?
\end{frage*}

Wir betrachten nun den Fall, in dem $m=p$ prim ist.

\begin{defn*}[Gauß-Summe]\index{Gauß-Summe}
	Sei $p$ eine Primzahl und $\chi$ ein Dirichlet-Charakter modulo $p$ mit $\chi \neq \chi_0$. Für $a \in \Z$ definieren wir die \emph{Gauß-Summen}
	\[ g_a(\chi) = \sum_{t=0}^{p-1} \chi(t) e\left( \frac{at}{p} \right), \]
	\[ g_a(\chi_0) = \sum_{t=0}^{p-1} e\left( \frac{at}{p} \right). \]
\end{defn*}

\begin{notat*}
	Für $a=1$ schreibe auch $g_1(\chi) = g(\chi)$.
\end{notat*}

\begin{rem*}
	\begin{enumerate}[label={\roman*})]
		\item Ist $a \equiv b \bmod p$, so gilt
			\[ g_a(\chi) = g_b(\chi) \]
			für alle Dirichlet-Charaktere modulo $p$.
		\item Für den Hauptcharakter $\chi_0$ berechnen wir
			\begin{align*}
				g_a(\chi_0) &= \sum_{t=0}^{p-1} e\left( \frac{at}{p} \right)\\
				&= \begin{cases}
					p\quad &p \mid a\\
					0 &p \nmid a
				\end{cases}
			\end{align*}
	\end{enumerate}
\end{rem*}

\begin{frage*}
	Wie funktioniert die Berechnung von Gauß-Summen für allgemeine Charaktere $\chi$?
\end{frage*}

Zur Beantwortung dieser Frage betrachten wir zunächst die Reduktion auf den Fall $a=1$, d.h. $g_1(\chi) = g(\chi)$.

\begin{lem}\autolabel
	Sei $p$ prim, $a \in \Z$ und $\chi$ ein Dirichlet-Charakter modulo $p$ mit $\chi \neq \chi_0$. Falls $p \nmid a$ sei $a^* \in \Z$ mit $aa^* \equiv 1 \bmod p$. Dann gilt
	\[ g_a(\chi) = \begin{cases}
		0 &p \mid a\\
		\chi(a^*)g(\chi) &p \nmid a
	\end{cases} \]
\end{lem}

\begin{frage*}
	Welche Größenordnung könnte man für $g(\chi)$ für große Primzahlen $p$ erwarten?
\end{frage*}

\begin{idee*}
	Falls die Summanden $\chi(t) e\left( \frac{t}{p} \right)$ zufällig auf dem Einheitskreis verteilt wären, so könnten wir für ein großes $p$
	\[ |g(\chi)| \sim \sqrt{p} \]
	erwarten (square root cancellation).
\end{idee*}

\begin{thm}\autolabel
	Sei $p$ prim und $\chi \neq \chi_0$ ein Dirichlet-Charakter modulo $p$. Dann gilt
	\[ |g(\chi)| = \sqrt{p}. \]
\end{thm}

Im Spezialfall, dass $\chi = \left(\frac{\cdot}{p}\right)$ das Legendre-Symbol ist, können wir noch etwas mehr sagen:

\begin{thm}\autolabel
	Sei $p>2$ prim und $\left(\frac{\cdot}{p}\right)$ das Legendre-Symbol. Dann gilt
	\[ g(\chi)^2 = (-1)^{\frac{p-1}{2}}p. \]
\end{thm}

Eine erste Anwendung von Gauß-Summen ist der Beweis des quadratischen Reziprozitätsgesetzes. eine weitere ist die

\subsection{Arithmetik in Kreisteilungskörpern}

Sei $p$ eine Primzahl und $\zeta = e\left( \frac{1}{p} \right) = e^{\frac{2\pi i}{p}}$. Dann ist
\[ \Q(\zeta) = \big\{ a_0 + a_1\zeta + \dots + a_{p-2}\zeta^{p-2} \mid a_0,a_1,\dotsc,a_{p-2} \in \Q \big\} \subset \C \]
ein Körper und
\[ \Z[\zeta] = \big\{ a_0 + a_1\zeta + \dots + a_{p-2}\zeta^{p-2} \mid a_0,a_1,\dotsc,a_{p-2} \in \Z \big\} \]
ein Teilring.

\begin{rem*}
	Es gilt
	\[ 1 + \zeta + \dots + \zeta^{p-1} = \sum_{t=0}^{p-1} e\left( \frac{t}{p} \right) = 0. \]
	Also können wir $\zeta^{p-1}$ schreiben als
	\[ \zeta^{p-1} = -1 - \zeta - \zeta^2 - \dots - \zeta^{p-2} \in \Z[\zeta]. \]
\end{rem*}

\begin{notat*}
	Sind $\alpha,\beta \in \Z[\zeta]$ und $q \in \Z$ eine weitere Primzahl, so schreiben wir 
	\[ \alpha \equiv \beta \bmod q \]
	falls $\alpha-\beta = q\gamma$ mit $\gamma \in \Z[\zeta]$ ist.
\end{notat*}

\begin{exmp*}
	$p=5,\ q = 7,\ 7 + \zeta + 8\zeta^2 \equiv \zeta + \zeta^2 \bmod 7$
\end{exmp*}

\begin{lem}\autolabel
	Sei $q \neq p$ eine weitere Primzahl und $m,n \in \Z$ mit $m \equiv n \bmod q$ im Ring $\Z[\zeta]$. Dann gilt bereits $m-n = q \cdot c$ mit einem $c \in \Z$.
\end{lem}