Um den Satz \ref{4.10} nun beweisen zu können bedarf es noch ein wenig Vorbereitung:\lecturevideo{27.04.2021}

\begin{thm}\autolabel
	Sei $P(X) = a_nX^n + \dots + a_1X + a_0$ mit $a_n,\dotsc,a_0 \in \Z$ und $p$ prim. Angenommen $p \nmid a_n$, dann hat die Gleichung $P(X) \equiv 0 \bmod p$ höchstens $n$ verschiedene Lösungen modulo $p$.
\end{thm}

Für welche $M \in \N_{\geq 2}$ gibt es eine Primitivwurzel modulo $M$?\video

\begin{thm}\autolabel
	Sei $k \in \N$ und $p > 2$ eine Primzahl. Dann gibt es eine Primitivwurzel modulo $p^k$.
\end{thm}

\begin{lem}\autolabel
	Sei $p$ eine ungerade Primzahl und $k \in \N$. Dann gilt
	\[ (1+p)^{p^{k-1}} \equiv 1+p^k \mod p^{k+1}. \]
\end{lem}