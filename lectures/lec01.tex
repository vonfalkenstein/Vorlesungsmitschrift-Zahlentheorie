\chapter[Primzahlen]{Primzahlen - Bausteine der ganzen Zahlen}\lecturefile{Primzahlen, Teilbarkeit}

%\[ \begin{tikzcd}
%	&&&G/\ker f \arrow{ddr}{\iota}&\\
%	\\
%	\ker f \arrow{rr}{\kappa}&& G \arrow{rr}{f} \arrow{uur}{\pi}& & H
%\end{tikzcd} \]

Wo liegen die Unterschiede zwischen $\Q$ und $\Z$?

\begin{exmp*}
	Seien $a,b \in \Z$, $a\neq 0$. Dann hat die Gleichung
	\[ ax = b \]
	nicht immer eine Lösung $x \in \Z$.
\end{exmp*}

\begin{defn*}[Teiler]\index{Teiler}
	Seien $a,b \in \Z$. Wir sagen, dass \emph{$\emph{a}$ ein Teiler von $\emph{b}$ ist} ($a \mid b$), falls es eine ganze Zahl $x \in \Z$ gibt mit $ax = b$.
\end{defn*}

\begin{lem}\autolabel
	Seien $a,b,c,d \in \Z$.
	\begin{enumerate}[label={\roman*})]
		\item Falls $d \mid a$ und $d \mid b$, dann $d \mid a+b$.
		\item Ist $d \mid a$, dann auch $d \mid ab$.
		\item Ist $d \mid a$, dann gilt $db \mid ab$.
		\item Gilt $d \mid a$ und $a \mid b$, dann $d \mid b$.
		\item Ist $a \neq 0$ und $d \mid a$, dann gilt $|d| \leq |a|$.
	\end{enumerate}
\end{lem}

\begin{rem*}
	Eine ganze Zahl $a \neq 0$ hat höchstens endlich viele Teiler.
\end{rem*}

\begin{thm}[Teilen mit Rest]\autolabel
	Seien $a,b \in \Z, b > 0$. Dann gibt es $q,r \in \Z$ mit
	\[ a = bq + r, \ 0 \leq r < b. \]
\end{thm}

\begin{defn*}[Primzahl]\index{Primzahl}
	Eine ganze Zahl $p > 1$, die genau zwei positive Teiler hat (1 und sich selbst), nenn wir \emph{Primzahl}.
\end{defn*}

\begin{exmp*}
	\( 2,3,5,7,11,13,17,19,23,29,31,\dotsc \)
\end{exmp*}

\begin{lem}\autolabel
	Sei $n \in \N, n > 1$ und sei $p>1$ der kleinste positive Teiler von $n$. Dann ist $p$ eine Primzahl. Ist außerdem $n$ nicht prim, dann gilt $p \leq \sqrt{n}$.
\end{lem}