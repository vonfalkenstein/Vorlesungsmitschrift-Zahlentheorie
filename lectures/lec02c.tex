\chapter{Kongruenzen}\video

\begin{exmp*}
	Die Uhr
	\incfig{3_1}{10cm}
	Der Minutenzeiger weiß nur, wie viele Minuten es nach einer vollen Stunde ist
\end{exmp*}

\begin{defn*}[Kongruenz, Kongruenzklasse]\index{Kongruenzklasse}
	Sei $M \in \N,\ M > 1,\ a,b \in \Z$. Wir sagen, dass \emph{$\emph{a}$ kongruent zu $\emph{b}$ ist modulo $\emph{M}$}, falls \( M \mid (a-b) \), schreibe \( a \equiv b \mod M \).\\
	Sei $r \in \Z$. Die Menge aller ganzen Zahlen $x$ mit $x \equiv r \mod M$ nennen wir die \emph{Kongruenzklasse von $\emph{r}$ modulo $\emph{M}$}.
\end{defn*}

\begin{exmp*}
	\( 7 \equiv  27 \mod 10,\ 4 \equiv 1 \mod 3 \)
\end{exmp*}

\begin{lem}\autolabel
	Sei $M > 1,\ M \in \N,\ a_1,a_2,b_1,b_2 \in \Z$ mit $a_1 \equiv a_2 \mod M$ und $b_1 \equiv b_2 \mod M$. Dann gilt
	\begin{enumerate}[label={\roman*})]
		\item $a_1 + b_1 \equiv a_2 + b_2 \mod M$
		\item $a_1 - b_1 \equiv a_2 - b_2 \mod M$
		\item $a_1b_1 \equiv a_2b_2 \mod M$
	\end{enumerate}
\end{lem}

\begin{notat*}
	Schreibe \( \Z / M\Z \) für die Menge aller Restklassen modulo $M$.
\end{notat*}

\emph{Eine Anwendung:} Eine natürliche Zahl $n$ ist durch 9 teilbar genau denn, wenn die Summe ihrer Ziffern in der Dezimaldarstellung (also ihre Quersumme) durch 9 teilbar ist.

\begin{exmp*}
	43227 ist durch 9 teilbar.
\end{exmp*}

\section*{Inverse Restklassen}\filevideo{Inverse Restklassen}

Seien \( x,y \in \Z \) mit $2x = 2y $. Die echten Detektive unter uns erkennen, dass man dies einfach zu $x = y$ kürzen kann. Nun ist die Frage, ob das auch funktioniert, wenn wir statt über $\Z$ über dem Restklassenring arbeiten, also ob wir auch bei Kongruenzen kürzen können.

\begin{exmp*}
	\begin{enumerate}
		\item[]
		\item \( 2x \equiv 2y \mod 4 \overset{?}{\implies} x \equiv y \mod 4 \)\\
			\textcolor{red}{Nein $\lightning$}, z.B. \( 2 \cdot 3 \equiv 2 \cdot 1 \mod 4,\ 3 \not\equiv 1 \mod 4 \)
		\item \( 2x \equiv 2y \mod 5 \implies 5 \mid (2x-y) \), d.h. \( 5 \mid x-y \implies x \equiv y \mod 5 \)\\
			\underline{oder} bemerke, dass \( 2 \cdot 3 \equiv 1 \mod 5 \) und multipliziere die obige Kongruenz mit 3.
	\end{enumerate}
\end{exmp*}

\begin{defn*}[Invertierbare Restklasse]\index{Kongruenzklasse!Invertierbare Restklasse}
	Sei \( M \in \N,\ M > 1 \). Wir nennen \( a \in \Z \) \emph{invertierbar modulo $\emph{M}$}, falls es \( \existss b \in \Z \) gibt mit \( ab \equiv 1 \mod M \). In dem Fall nennen wir die Restklasse \( a\) (modulo $M$) invertierbar.
\end{defn*}

\begin{exmp*}
	2 ist invertierbar modulo 25, denn \( 2 \cdot 13 \equiv 1 \mod 25 \).
\end{exmp*}

\begin{thm}\autolabel
	Sei \( M \in \N,\ M > 1,\ a \in \Z \). Dann ist $a$ invertierbar modulo $M$ genau dann, wenn \( \ggt(a,M) = 1 \).
\end{thm}

\begin{rem*}
	Ist \( a \in \Z \) invertierbar modulo $M$, dann ist jedes Element in der Restklasse $a \mod M$ invertierbar modulo $M$.\video Die Menge aller $b \in \Z$ mit $ba \equiv 1 \mod M$ ist eine Restklasse modulo $M$, schreibe $a^{-1}$ (modulo $M$) für diese Restklasse.
\end{rem*}

Sei \( a \in \Z,\ M \in \N,\ M > 1 \) mit \( \ggt(a,M) = 1 \). Wie können wir die inverse Restklasse $a^{-1}$ berechnen?\\
$\implies$ wir verwenden den erweiterten Euklidischen Algorithmus um \( x,y \in \Z \) zu finden mit \( ax+My = 1 \).

\begin{notat*}
	Ist $M > 1$, so schreiben wir \( (\Z / M\Z)^* \) für die Menge der invertierbaren Restklassen modulo $M$.
\end{notat*}

\begin{exmp*}
	\begin{enumerate}
		\item[]
		\item \( (\Z/6\Z)^* = \left\{\bar{1},\bar{5}\right\} \), also \( |(\Z/6\Z)^*| = 2 \)
		\item Sei $p$ prim.
		\[ (\Z/p\Z)^* = \{1, 2, \dotsc, p-1 \} \implies |(\Z/p\Z)^*| = p-1 \]
	\end{enumerate}
\end{exmp*}

\begin{lem}\autolabel
	Sei $p$ prim. Dann gilt \( (p-1)! \equiv -1 \mod p. \)
\end{lem}
