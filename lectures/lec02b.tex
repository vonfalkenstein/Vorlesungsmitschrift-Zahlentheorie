\chapter{Die Teilerfunktion} \filevideo{Teilerfunktion, Kongruenzen}

\begin{defn*}[Teilerfunktion]\index{Teilerfunktion}
	Sei $n \in \N$. Wir definieren \( d(n) \) als die Zahl der positiven Teiler von $n$, d.h.
	\[ d(n) = \sum_{\substack{d \mid n \\ d \geq 1}} 1. \]
	Weiter definieren wir
	\[ S(n) = \sum_{\substack{d \mid n \\ d \geq 1 \\ d < n}} d \]
	und 
	\[ \sigma(n) = \sum_{\substack{d \mid n \\ d \geq 1}} d \]
\end{defn*}

\begin{exmp*}
	\( d(7) = 2,\ d(6) = 4 \)\\
	Sei $p$ eine Primzahl, dann gilt $d(p) = 2$ und $d(p^k) = k+1$ für $k\geq 0$.
\end{exmp*}

\begin{rem*}
	$\sigma(n) = S(n) + n$
\end{rem*}

\begin{defn*}[perfekte Zahl] \index{perfekte Zahl}
	Wir nennen eine natürliche Zahl $n$ \emph{perfekt}, falls
	\[ S(n) = n. \]
\end{defn*}

\begin{exmp*}
	$6 = 1+2+3$ ist perfekt. 28 ist perfekt.
\end{exmp*}

\begin{lem}\autolabel
	Seien $m,n \in \N$ mit $\ggt(m,n) = 1$. Dann gilt
	\[ d(mn) = d(m)d(n) \]
	und \[ \sigma(mn) = \sigma(m) \sigma(n). \]
\end{lem}

\begin{rem*}
	$d(n)$ und $\sigma(n)$ sind sogenannte \textit{multiplikative Funktionen}.
\end{rem*}

Kennt man die Primfaktorzerlegung von $n$, so lassen sich diese Funktionen sehr einfach berechnen. Wir wollen nun eine allgemeine Formel für $d(n)$ aufstellen.

Sei $n = p_1^{k_1} \dotsm p_r^{k_r}$ mit $p_1 < \dots < p_3$ Primzahlen, $k_1, \dotsm, k_r \geq 0$. Dann gilt nach Lemma \ref{2.1}
\begin{align*}
	d(n) &= d \left( p_1^{k_1} \dotsm p_r^{k_r} \right)\\
	&= d\left(p_1^{k_1}\right) \dotsm d\left(p_r^{k_r}\right)
\end{align*}
Es gilt
\[ d(p^k) = k + 1, \]
also
\[ d(n) = (k_1+1) (k_2+1) \dotsm (k_r+1). \]
Weiterhin berechnen wir
\begin{align*}
	\sigma(n) &= \sigma\left(p_1^{k_1}\right) \dotsm \sigma\left(p_r^{k_r}\right)\\
	\sigma\left(p^k\right) &= 1 +p + p^2 + \dots + p^k\\
	&= \frac{p^{k+1} -1}{p-1}
\end{align*}
Wir erhalten
\[ \sigma(n) = \frac{p_1^{k_1+1} - 1}{p_1-1} \dotsm \frac{p_r^{k_r+1}-1}{p_r-1} \]

\begin{thm}\autolabel
	Sei $n = p_1^{k_1} \dotsm p_r^{k_r}$ mit $p_1 < \dots < p_3$ Primzahlen, $k_1, \dotsm, k_r \geq 0$. Dann gilt 
	\begin{align*}
		d(n) &= \prod_{i=1}^r (k_i+1)\\
		\sigma(n) &= \prod_{i=1}^r \frac{p_i^{k_i + 1}-1}{p_i-1}.
	\end{align*}
\end{thm}

\begin{exmp*}
	\( d(25 \cdot 3) = d(25) d(3) = d(5^2) d(3) = 3 \cdot 2 = 6 \)
\end{exmp*}

\begin{rem*}
	$S(n)$ ist keine multiplikative Funktion:\\
	\( 1 = S(2)S(3) \neq S(6) = 6 \)
\end{rem*}