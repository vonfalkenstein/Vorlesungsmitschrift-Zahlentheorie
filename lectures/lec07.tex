\begin{thm}\autolabel
	\lecture{04.05.2021}\addtocounter{video}{1}Sei $p$ eine Primzahl mit $p \equiv -1 \bmod 4$ und wir nehmen an, dass $2p + 1$ ebenfalls prim ist. Dann gilt
	\[ 2p+1 \mid 2^p -1.\video{} \]
\end{thm}

Dieser Satz sagt insbesondere, dass für $p>3$ in diesem Fall $2^p-1$ keine Primzahl ist. Er vor allem relevant in der noch offenen Frage, ob es unendlich viele Mersenne\footnote{Nach Marin Mersenne (1588-1648), ein französischer Theologe, Mathematiker und Musiktheoretiker}-Primzahlen (also Primzahlen der Form $2^p-1$) gibt.

\begin{thm}[Pépin\protect\footnotemark-Test]\autolabel
	\footnotetext{Nach Théophile Pépin (1826-1904), ein französischer Mathematiker}
	Definiere $F_n := 2^{2^n}+1$ für $n \in \N$. Dann gilt, dass $F_n$ genau dann prim ist, wenn gilt
	\[ 3^{\frac{1}{2}(F_n-1)} \equiv -1 \mod F_n. \]
\end{thm}

Die im Satz \ref{6.11} definierten Zahlen $F_n$ werden Fermat-Zahlen genannt. Bisher ist noch unbekannt, ob unendlich viele der Fermat-Zahlen auch Primzahlen sind. Fermat hatte seinerzeit behauptet, alle Fermat-Zahlen seien Primzahlen, dies wurde allerdings von Euler widerlegt, der zeigte, dass $F_5$ nicht prim ist. Inzwischen wird angenommen, dass tatsächlich lediglich $F_0,\dotsc,F_4$ prim sind. Da die Fermat-Zahlen allerdings durch die doppelte Exponentialität so dünn sind ist es sehr schwer, sie tatsächlich zu überprüfen. Die kleinste bisher ungetestete Fermat-Zahl ist $F_{33}$, welche 2.585.287.973 Stellen hat.

\begin{lem}[Gauß'sches Lemma]\autolabel
	Sei\filevideo{Quadratische Reziprozität} $p>2$ eine Primzahl. Wir nennen \( 1,2,\dotsc,\frac{p-1}{2} \bmod p\) positive Restklassen und $-1,-2,\dotsc,-\frac{p-1}{2} \bmod p$ negative Restklassen $\bmod p$. Sei $a \in \Z$ mit $p \nmid a$ und $\mu$ die Zahl der negativen Restklassen in der Folge $a,2a,\dotsc,\frac{p-1}{2}a \bmod p$. Dann gilt
	\[ \left(\frac{a}{p}\right) = (-1)^\mu. \]
\end{lem}

\begin{exmp*}
	$\left(\frac{3}{5}\right)$:\\
	$1,2$ positive Restklassen modulo 5\\
	$3,4$ negative Restklassen modulo 5\\
	$3 \cdot 1 \equiv -2,\ 3\cdot 2 \equiv 1$. Das heißt in diesem Beispiel haben wir $\mu = 1$ und $\left(\frac{3}{5}\right) = (-1)^\mu = -1$.
\end{exmp*}

\addtocounter{video}{1}
\begin{defn*}[Untere Gaußklammer]
	Sei\video{} $x \in \R$. Wir definieren
	\[ \lfloor x \rfloor = \max\{m \in \Z\mid m \leq x\}. \]
\end{defn*}

\begin{exmp*}
	\( \lfloor 3 \rfloor = 3,\ \lfloor2,8\rfloor = 2 \)
\end{exmp*}

\begin{lem}\autolabel
	Sei $p>2$ eine Primzahl, $a \in \Z$ mit $2 \nmid a$ und $p \nmid a$. Sei 
	\[ S(a,p) := \sum_{s = 1}^{\frac{p-1}{2}} \left\lfloor \frac{as}{p}\right\rfloor. \]
	Dann gilt
	\[ \left(\frac{a}{p}\right) = (-1)^{S(a,p)}. \]
\end{lem}