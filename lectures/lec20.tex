\section{Zählen von Lösungen von Kongruenzgleichungen}\lecture{18.06.2021}

\begin{frage*}
	Sei $p$ eine Primzahl und betrachte die Gleichung
	\[ x^n + y^n \equiv 1 \bmod p. \]
	Wie viele Lösungen hat diese Kongruenz?
\end{frage*}

Wir betrachten zunächst den Fall $n = 2$. Sei $N(p)$ die Zahl der Lösungen der Kongruenz 
\[ x^2+y^2 \equiv 1 \bmod p. \]

\begin{idee*}
	Schreibe $N(p)$ als Summe von Charaktersummen.
\end{idee*}

Sei $p>2$ und $\left(\frac{\cdot}{p}\right)$ das Legendre-Symbol modulo $p$. Ist $a \in \Z$, so gilt
\[ 1 + \left(\frac{a}{p}\right) = \left| \{ y \bmod p \mid y^2 \equiv a \bmod p\} \right|. \]
Schreibe $N(p)$ als
\begin{align*}
	N(p) &= \sum_{\substack{0 \leq a, b \leq p-1\\a + b \equiv 1 \bmod p}} \left( 1 + \left(\frac{a}{p}\right) \right) \left( 1 + \left(\frac{b}{p}\right) \right)\\
	&= \sum_{\substack{0 \leq a, b \leq p-1\\a + b \equiv 1 \bmod p}} \left( 1 + \left(\frac{a}{p}\right) \left(\frac{b}{p}\right) \left(\frac{ab}{p}\right) \right)\\
	&= p + \sum_{\substack{0 \leq a, b \leq p-1\\a + b \equiv 1 \bmod p}} \left(\frac{a}{p}\right) + \sum_{\substack{0 \leq a, b \leq p-1\\a + b \equiv 1 \bmod p}} \left(\frac{b}{p}\right) + \sum_{\substack{0 \leq a, b \leq p-1\\a + b \equiv 1 \bmod p}} \left(\frac{a}{p}\right)\left(\frac{b}{p}\right)
\end{align*}
Es ist
\begin{align*}
	\sum_{\substack{0 \leq a, b \leq p-1\\a + b \equiv 1 \bmod p}} \left(\frac{a}{p}\right) &= \sum_{0 \leq a \leq p-1} \left(\frac{a}{p}\right)\\
	&= g_0 \left( \left(\frac{\cdot}{p}\right) \right)\\
	&\overset{\ref{12.5}}{=} 0 
\end{align*}
Genauso gilt
\[ \sum_{\substack{0 \leq a, b \leq p-1\\a + b \equiv 1 \bmod p}} \left(\frac{b}{p}\right) = 0 \]
Es folgt
\[ N(p) = p + \sum_{\substack{0 \leq a, b \leq p-1\\a + b \equiv 1 \bmod p}} \left(\frac{a}{p}\right)\left(\frac{b}{p}\right) \]

\begin{defn*}[Jacobi-Summe]\index{Jacobi-Summe}
	Sei $p$ prim und seien $\chi,\psi$ Dirichlet-Charaktere modulo $p$, ungleich dem Hauptcharakter. Dann definieren wir die Jacobi-Summe $J(\chi,\psi)$ durch
	\[ J(\chi,\psi) = \sum_{\substack{0 \leq a, b \leq p-1\\a + b \equiv 1 \bmod p}} \chi(a) \psi(b). \]
	Für den Hauptcharakter $\chi_0$ setzen wir
	\[ J(\chi_0,\chi) = J(\chi,\chi_0) = 0 \]
	und $J(\chi_0,\chi_0) = p$.
\end{defn*}

\begin{thm}\autolabel
	Sei $p$ prim und $\chi,\psi$ Dirichlet-Charaktere modulo $p$, ungleich dem Hauptcharakter $\chi_0$. Dann gilt
	\begin{enumerate}[label={\roman*})]
		\item $J(\chi,\bar{\chi}) = -\chi(-1)$
		\item Ist $\chi\psi \neq \chi_0$, so ist
			\[ J(\chi,\psi) = \frac{g(\chi)g(\psi)}{g(\chi\psi)} \]
			und $|J(\chi,\psi)| = \sqrt{p}$.
	\end{enumerate}
\end{thm}

Dies wollen wir nun auf die Zählfunktion anwenden, also auf
\[ N(p) = |\{0 \leq x,y \leq p-1 \mid x^2+y^2 \equiv 1 \bmod p\} \]
Wir haben bereits berechnet, dass für $\chi = \left(\frac{\cdot}{p}\right)$
\[ N(p) = p + J(\chi,\chi) \]
gilt. Nach Satz \ref{12.9} gilt wegen $\chi = \bar{\chi}$ $N(p) = p-\left(\frac{-1}{p}\right)$, also
\[ N(p) = \begin{cases}
	p-1 &p \equiv 1 \bmod 4\\
	p+2 &p \equiv 3 \bmod 4
\end{cases} \]

Als eine weitere Anwendung können wir die Zählfunktion
\[ N_3(p) = |\{0 \leq x,y \leq p-1 \mid x^3+y^3 \equiv 1 \bmod p\} \]
für $p>3$ prim studieren.\\
\emph{Fall 1:} $p \equiv 2 \bmod 3$. Dann ist
\[ |(\Z/p\Z)^*| = p-1 \equiv 1 \bmod 3 \]
und die Abbildung
\[ \alpha: (\Z/p\Z)^* \to (\Z/p\Z)^*,\quad x \mapsto x^3 \]
bijektiv.\\
\emph{Fall 2:} $p \equiv 1 \bmod 3$. Sei $g$ eine Primitivwurzel modulo $p$ und definieren den Dirichlet-Charakter $\chi$ modulo $p$ durch
\[ \chi(a) = \begin{cases}
	0 &p \mid a\\
	e\left( \frac{c}{3} \right) &a \equiv g^c \bmod p
\end{cases} \]
Dann gilt für $a \in \Z$
\[ |\{0 \leq x \leq p-1 \mid x^3 \equiv a \bmod p\}| = 1 + \chi(a) + \chi^2(a). \]
Damit berechnen wir
\begin{align*}
	N_3(p) &= \sum_{\substack{0 \leq a, b \leq p-1\\a + b \equiv 1 \bmod p}} |\{ 0 \leq x \leq p-1 \mid x^3 \equiv a \bmod p \}| \cdot |\{ 0 \leq x \leq p-1 \mid y^3 \equiv a \bmod p \}|\\
	&= \sum_{\substack{0 \leq a, b \leq p-1\\a + b \equiv 1 \bmod p}} (1 + \chi(a) + \chi^2(a))(1 + \chi(b) + \chi^2(b))\\
	&= \sum_{\substack{0 \leq a, b \leq p-1\\a + b \equiv 1 \bmod p}} (1 + \chi(a) + \chi^2(a) + \chi(b) + \chi(a)\chi(b) + \dots + \chi^2(a)\chi^2(b))\\
	&= p + J(\chi,\chi) + J(\chi^2,\chi) + J(\chi,\chi^2) + J(\chi^2,\chi^2)\\
	&= p - \chi^2(-1) - \chi(-1) + J(\chi,\chi) + J(\chi^2,\chi^)
\end{align*}

\begin{rem*}
	$-1 \equiv g^{\frac{p-1}{2}} \bmod p$, also $\chi(-1) = e\left( \frac{p-1}{6} \right) = 1$, da $6 \mid p-1$ und ebenso $\chi^2(-1) = 1$.
\end{rem*}

Insgesamt erhalten wir
\[ N_3(p) = p-2 + J(\chi,\chi) + J(\chi^2,\chi^2). \]
Nach Satz \ref{12.9} gilt
\[ |N_3(p) - (p-2)| \leq 2 \sqrt{p}. \]

\begin{rem*}
	Für $p$ hinreichend groß, genauer $p-2 > 2\sqrt{p},$ folgt $N_3 \neq 0$ und damit hat die Gleichung $x^3+y^3 \equiv 1 \bmod p$ eine Lösung.
\end{rem*}