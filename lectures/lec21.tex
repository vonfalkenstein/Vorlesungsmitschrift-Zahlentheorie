\chapter{Multiplikative Funktionen}\lecture{22.06.2021}

\begin{exmp*}
	Die Teilerfunktion
	\[ d(n) = \sum_{d \mid n} 1 \]
	hat die Eigenschaft
	\[ d(nm) = d(n)d(m) \quad \foralll m,n: \ggt(m,n)=1. \]
\end{exmp*}

\begin{defn*}[arithmetische Funktion]\index{arithmetische Funktion}
	\begin{itemize}
		\item []
		\item Wir nennen eine Abbildung
			\[ f: \N \to \C \]
			eine \emph{arithmetische Funktion}.
		\item Eine arithmetische Funktion $f: \N \to \C$ heißt \emph{multiplikativ}, wenn
			\[ f(nm) = f(n)f(m) \quad \foralll m,n \in \N: \ggt(m,n)=1. \]
		\item Eine arithmetische Funktion $f: \N \to \C$ heißt \emph{vollständig multiplikativ}, wenn
			\[ f(nm) = f(n)f(m) \quad \foralll m,n \in \N. \] 
	\end{itemize}
\end{defn*}

\begin{rem*}
	Ist $f \neq 0$ eine multiplikative Funktion, so gilt $f(1)=1$.
\end{rem*}

\begin{exmp*}
	\begin{enumerate}[label={\roman*})]
		\item[]
		\item Die Teilerfunktion ist multiplikativ, aber nicht vollständig multiplikativ.
		\item Die Funktion \( \sigma(n) = \sum_{d \mid n} d \) ist multiplikativ.
		\item Definiere \[ \mathds{1}: \N \to \C \quad n \mapsto 1. \] $\mathds{1}$ ist vollständig multiplikativ.
		\item \[ \epsilon: \N \to \C,\quad n \mapsto \begin{cases}
			1 &n=1\\0 &n\neq1
		\end{cases} \] ist vollständig multiplikativ.
		\item Für $k \in \N$ sind die Funktionen $n \mapsto n^k$ vollständig multiplikativ. Für $k=1$ ist das einfach $\id: N \to \C,\ n \mapsto n$.
		\item Die Eulersche $\varphi$-Funktion ist multiplikativ.
	\end{enumerate}
\end{exmp*}

\begin{obs*}
	Die Funktion
	\[ \sigma(n) = \sum_{d \mid n} d = \sum_{d \mid n} d \cdot \mathds{1} \left(\frac{n}{d}\right) \]
	ist "zusammengesetzt" aus den Funktionen $\id$ und $\mathds{1}$.
\end{obs*}

\begin{defn*}[Faltung]\index{arithmetische Funktion!Faltung}
	Seien $f,g: \N \to \C$ arithmetische Funktionen. Wir definieren die \emph{Faltung} von $f$ und $g$ durch
	\[ f * g(n) = \sum_{d \mid n} f(d) g(\left(\frac{n}{d}\right)). \]
\end{defn*}

\begin{exmp*}
	$\sigma = \id * \mathds{1}$\\$d = \mathds{1} * \mathds{1}$
\end{exmp*}

\begin{lem}\autolabel
	Seien $f,g,h: \N\to \C$ arithmetische Funktionen. Dann gilt
	\begin{enumerate}[label={\roman*})]
		\item $f * g = g * f$
		\item $f*(g*h) = (f*g)*h$
		\item $\epsilon * f = f * \epsilon = f$
		\item Sind $f$ und $g$ multiplikativ, so ist auch $f*g$ multiplikativ.
	\end{enumerate}
\end{lem}

\begin{rem*}
	Arithmetische Funktionen bilden unter punktweiser Addition und der Faltung $*$ als Multiplikation einen kommutativen Ring.
\end{rem*}

\begin{frage*}
	Bestimme das "Inverse zu $\mathds{1}$ unter der Faltung", d.h. finde eine arithmetische Funktion $g: \N \to \C$ mit $1 * g = \epsilon$.
\end{frage*}

$\to$ Berechne $\mathds{1} * g(1) = 1 \implies g(1) = 1$\\
Für eine Primzahl $p$ ist
\begin{align*}
	g(1) + g(p) = 1 * g(p) = \epsilon(p) = 0 \implies g(p) = -1
\end{align*}
Für Primzahlpotenzen $p^k,\ k \in \N$, gilt
\[ g(1) + g(p) + \dots + g(p^k) = \mathds{1} * g(p^k) = 0. \]
Induktiv über $k$ folgt $g(p^k)=0$ für $p$ prim, $k\geq 2$.

\begin{defn*}[Möbius-Funktion]\index{Möbius-Funktion}
	Wir definieren die \emph{Möbius-Funktion}
	\[ \mu: \N \to \{0,+1,-1\} \]
	durch
	\[ \mu(n) = \begin{cases}
		1 &n=1\\
		(-1)^r &n = p_1 \dotsm p_r\\
		0 &\text{sonst}
	\end{cases} \]
\end{defn*}

\begin{lem}\autolabel
	Die Möbius-Funktion $\mu$ ist multiplikativ und es gilt\[ \mu * \mathds{1} = \epsilon. \]
\end{lem}

\begin{cor}\autolabel
	Seien $F,f: \N \to \C$ arithmetische Funktionen mit $F=f*\mathds{1}$, d.h.
	\[ F(n) = \sum_{d \mid n} f(d). \]
	Dann gilt
	\[ f(n) = \sum_{d \mid n} F(d) \mu \left(\frac{n}{d}\right). \]
\end{cor}

\begin{rem*}
	Ist $F=f*\mathds{1}$ und $F$ multiplikativ, so folgt aus Korollar \ref{13.3}, dass auch $f$ multiplikativ ist.
\end{rem*}

\begin{lem}\autolabel
	Sei $\varphi$ die Eulersche $\varphi$-Funktion. Dann gilt
	\[ \varphi * \mathds{1} = \id. \]
\end{lem}

\begin{cor}\autolabel
	Die Eulersche $\varphi$-Funktion ist multiplikativ und es gilt
	\[ \varphi(n) = \sum_{d \mid n} d \mu \left(\frac{n}{d}\right). \]
\end{cor}