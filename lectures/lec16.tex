\section{Die $abc$-Vermutung für Polynomringe}
\lecture{04.06.2021}
\begin{idee*}
	Ersetze den Ring der ganzen Zahlen $\Z$ durch einen anderen Ring. Zunächst erstmal Polynomringe, z.B. $\Q[x]$. Wir studieren hier die Gleichung $a+b=c$.
\end{idee*}

Sei $F$ im Folgenden gleich $\Q,\C$ oder $\R$. Studiere den Ring $F[x]$.

\begin{rem*}
	Es gibt in $F[x]$ eine eindeutige Primfaktorzerlegung:\\
	Sei $f \in F[x],\ f \neq 0$. Dann können wir $f$ als Produkt schreiben
	\[ f = \phi P_1^{k_1} \cdot P_2^{k_2} \dotsm P_r^{k_r} \]
	mit $P_1,\dotsc,P_r$ paarweise verschiedene irreduzible\footnote{$g \in F[x]$ ist irreduzibel, falls aus $g = h \cdot p$ mit $h,p \in F[x]$ folgt $h \in F$ oder $p \in F$.} normierte\footnote{Das heißt der führende Koeffizient ist 1.} Polynome, $\phi \in F$ und $k_1,\dotsc,k_r \in \N,\ r \geq 0$.\\
	Diese Darstellung ist bis auf Reihenfolge der Faktoren eindeutig.
\end{rem*}

\begin{defn*}[Radikal eines Polynoms]\index{Radikal!eines Polynoms}
	Sei $f \in F[x] \setminus\{0\}$ gegeben in der Primfaktorzerlegung wie oben. Dann definieren wir das Radikal von $f$ als
	\[ \rad(f) := P_1 \cdot P_2 \dotsm P_r. \]
\end{defn*}

\begin{exmp*}
	\begin{enumerate}[label={\roman*})]
		\item $F=\Q,\ f = x^2+2x+1 = (x+1)^2$. Dann ist \[ \rad(f) = x+1. \]
		\item $F = \C,\ f = x^2+1$. Dann ist $f = (x+i)(x-i)$ und \[ \rad(f) = (x+i)(x-i) = f. \]
	\end{enumerate}
\end{exmp*}

\begin{defn*}
	Sind $f,g \in F[x]$ mit Primfaktorzerlegungen
	\begin{align*}
		f &= \phi_f \prod_{i=1}^n P_i^{k_i}\\
		g &= \phi_g \prod_{i=1}^n P_i^{l_i}
	\end{align*}
	mit $P_1,\dotsc,P_n$ paarweise verschiedene irreduzible normierte Polynome und $k_i,l_i \geq 0$. Dann definieren wir
	\[ \ggt(f,g) = \prod_{i=1}^n P_i^{\min(k_i,l_i)}. \]
\end{defn*}

\begin{lem}\autolabel
	Sei $f \in F[x]$ mit Primfaktorzerlegung
	\[ f = \phi \prod_{i=1}^r P_i^{k_i}, \]
	$\phi \in F,\ P_1,\dotsc,P_r$ paarweise verschiedene irreduzible normierte Polynome, $k_1,\dotsc,k_r \in \N$. Dann gilt
	\begin{enumerate}[label={\roman*})]
		\item $\ggt(f,f') = P_1^{k_1-1} \dotsm P_r^{k_r-1}$.\\
		\item $\phi \rad(f) = \frac{f}{\ggt(f,f')}$.
	\end{enumerate}
\end{lem}

\begin{thm*}[Die $ABC$-Vermutung für Polynome]
	Seien $A,B,C \in F[x]$, nicht alle konstant, mit $A+B=C$ und $\ggt(A,B,C) = 1$. Dann gilt
	\[ \max\big\{ \grad(A),\grad(B),\grad(C) \big\} \leq \grad\big(\rad(A,B,C)\big)-1. \]
\end{thm*}