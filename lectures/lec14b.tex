\chapter{Die $abc$-Vermutung}

1986: Masser\footnote{David Masser (geb. 1948), ein britischer Mathematiker} und Oesterlé\footnote{Joseph Oesterlé (geb. 1954), ein französischer Mathematiker} studieren die Gleichung
\[ a+b=c \]
in $a,b,c \in \N$.

\begin{obs*}
	\begin{itemize}
		\item Werden $a$ und $b$ durch hohe Potenzen von kleinen Primzahlen geteilt, dann gilt das für $c$ nicht.
		\item Sei $p_1,\dotsc,p_r$ die Menge von Primzahlen, die eine der Zahlen $a,b,c$ teilen. Dann ist $p_1 \dotsm p_r$ relativ groß im Vergleich zu $\max\{|a|,|b|,|c|\}$.
	\end{itemize}
\end{obs*}

\begin{exmp*}
	\begin{enumerate}[label={\roman*})]
		\item $2^4+3^3 = 16+27=43 \implies p_1p_2p_3 = 6 \cdot 43$
		\item Selbst experimentieren!
	\end{enumerate}
\end{exmp*}

\begin{defn*}[Radikal einer Zahl] \index{Radikal!einer Zahl}
	Sei $n \in \N$ mit Primfaktorzerlegung
	\[ n = p_1^{k_1} p_2^{k_2} \dotsm p_r^{k_r} \]
	mit $p_1 < p_2 < \dots < p_r$. Dann definieren wir das \emph{Radikal} von $n$ als 
	\[ \rad(n):= p_1 \dotsm p_r. \]
\end{defn*}

\begin{conj*}[$abc$-Vermutung]
	Seien $a,b,c \in \N$ mit $a+b=c$ und $\ggt(a,b,c)=1$. Dann gibt es für jedes $\epsilon > 0$ eine Konstante $C_0(\epsilon) > 0$ mit folgender Eigenschaft:\\
	Ist $c > C_0(\epsilon)$, dann ist 
	$$c < \rad(abc)^{1+\epsilon}.$$
\end{conj*}