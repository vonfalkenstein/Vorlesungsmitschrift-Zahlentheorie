\section{Binäre quadratische Formen}\lecture{02.07.2021}

\begin{defn*}[Binäre quadratische Form]\index{Quadratische Form!binäre}\index{Diskriminante}
	Eine \emph{binäre quadratische Form} ist eine Funktion $f: \Z^2 \to \Z$ der Form
	\[ (x,y) \mapsto ax^2 + bxy + cy^2 \]
	mit $a,b,c \in \Z$. Wir nennen \[ \begin{pmatrix}
		a & \frac{b}{2} \\ \frac{b}{2} & c
	\end{pmatrix} \] die zu $f$ \emph{assoziierte Matrix} und definieren die \emph{Diskriminante} von $f$ als
	\[ d(f) = b^2-4ac. \]
\end{defn*}

\begin{rem*}
	\begin{enumerate}[label={\roman*})]
		\item Es ist $d(f) = -4 D(f)$.
		\item Für jede binäre quadratische Form gilt $d(f) \equiv 0,1 \bmod 4$.
	\end{enumerate}
\end{rem*}

\begin{exmp*}
	$f(x,y) = x^2+y^2$ hat Diskriminante $d(f) = -4$.
\end{exmp*}

\begin{lem}\autolabel
	Sei $f(x,y) = ax^2+bxy+cy^2$ eine binäre quadratische Form mit $a>0$. Dann ist $f$ genau dann positiv definit, wenn $d(f)<0$.
\end{lem}

\begin{lem}\autolabel
	Sei $f$ eine binäre quadratische Form der Diskriminante $d(f)$. Dann ist $f$ genau dann ein Produkt von linearen Formen über $\Z$, wenn $d(f) = u^2$ für ein $u \in \Z$.
\end{lem}

\begin{frage*}
	Betrachte eine binäre quadratische Form $f(x,y)$ der Diskriminante $d$. Wie können wir eine möglichst "einfache" binäre quadratische Form $g$ finden mit $f \sim g$, also z.B. eine Form $g$ mit möglichst kleinen Koeffizienten?
\end{frage*}

\begin{lem}\autolabel
	Sei $f(x,y)$ eine binäre quadratische Form der Diskriminante $d \in \Z$, nicht negativ definit und $d \neq \square$. Dann gibt es eine binäre quadratische Form $g(x,y) = ax^2+bxy+cy^2$, die zu $f$ äquivalent ist und sodass gilt
	\begin{enumerate}[label={\roman*})]
		\item $|b| \leq |a| \leq |c|$.
		\item Ist $d>0$, so gilt $ac < 0$ und $|a| \leq \frac{1}{2} \sqrt{d}$.
		\item Ist $d < 0$, so gilt $c > 0$ und $0 < a \leq \sqrt{\frac{|d|}{3}}$.
	\end{enumerate}
\end{lem}