\section{Das Jacobi-Symbol}\lecturevideo{07.05.2021}

\emph{Motivation:} schnelle Berechnung von Legendre Symbolen. Angenommen $p$ ist eine (große) Primzahl, $n < p$ und wir wollen $\left(\frac{n}{p}\right)$ berechnen. Unsere Strategie sah dafür sah bisher wie folgt aus:

Nehme eine Primfaktorzerlegung von $n = q_1 \cdot q_2 \dotsm q_r$ mit $q_1,\dotsc,q_r$ prim. Dann gilt
\[ \left(\frac{n}{p}\right) = \left(\frac{q_1}{p}\right) \left(\frac{q_2}{p}\right) \dotsm \left(\frac{q_r}{p}\right) \]
Wende jetzt quadratische Reziprozität an.

\emph{Problem:} Wir müssen für dieses Vorgehen zunächst eine Primfaktorzerlegung von $n$ finden.

\begin{defn*}[Jacobi-Symbol]\index{Jacobi-Symbol}
	Sei $n \in \N$ ungerade und $m \in \Z$ mit $\ggt(m,n)=1$. Sei $n = p_1 p_2 \dotsm p_r$ mit $p_1,\dotsc p_r$ Primzahlen. Dann definieren wir
	\[ \left(\frac{m}{n}\right) = \left(\frac{m}{p_1}\right) \left(\frac{m}{p_2}\right) \dotsm \left(\frac{m}{p_r}\right). \]
\end{defn*}

\emph{Achtung:} Ist $\left(\frac{m}{n}\right)=1$, dann wissen wir im Allgemeinen noch nicht, ob die Kongruenz $m \equiv x^2 \bmod n$ eine Lösung hat!

\begin{exmp*}
	\begin{align*}
		\left(\frac{-1}{21}\right) &= \left(\frac{-1}{7}\right) \left(\frac{-1}{3}\right)\\
		&= (-1)(-1)\\
		&=1,
	\end{align*}
	aber $x^2 \equiv -1 \bmod 21$ hat keine Lösung!
\end{exmp*}

\begin{thm}[Eigenschaften von Jacobi-Symbolen]\autolabel
	Seien $m,n \in \N$ ungerade mit $\ggt(m,n)=1$. Dann gilt
	\begin{enumerate}[label={\roman*})]
		\item $\left(\frac{-1}{n}\right) = (-1)^\frac{n-1}{2}$
		\item \( \left(\frac{2}{n}\right) = (-1)^\frac{n^2}-1{8} \)
		\item \( \left(\frac{m}{n}\right) \left(\frac{n}{m}\right) = (-1)^\frac{(m-1)(n-1)}{4} \)
	\end{enumerate}
\end{thm}