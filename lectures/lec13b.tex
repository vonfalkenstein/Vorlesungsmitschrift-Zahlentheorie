\chapter{Die Pell'sche Gleichung}

Unser Ziel ist es, für $N \in \N,\ N \neq \square$, die Gleichung
\[ x^2-Ny^2=1 \]
zu studieren und (alle) Lösungen $(x,y)\in \Z_{\geq0}^2$ zu bestimmen.

\begin{exmp*}
	$N=2$, wir betrachten also $x^2-2y^2=1$. Als Lösungen haben wir beispielsweise $(1,0),(3,2),(17,12),\dotsc$.\\
	Angenommen $p,q \in \N$ mit $p^2-2q^2=1$. Dann ist $\frac{p^2}{q^2} - 2 = \frac{1}{q^2}$, also 
	\[ \frac{p^2}{q^2} \sim 2 \]
	und $\frac{p}{q}$ ist eine gute Näherung für $\sqrt{2}$.
\end{exmp*}

\begin{thm}\autolabel
	Sei $N \in \N,\ N \neq \square$, und $A \in \Z$ mit $|A| < \sqrt{N}$. Angenommen, $x,y \in \N$ erfüllen die Gleichung 
	\[ y^2-Ny^2=A. \]
	Dann ist $\frac{x}{y}$ ein Teilbruch vom Kettenbruch von $\sqrt{N}$.
\end{thm}

\begin{notat*}
	Schreibe $\frac{p_n}{q_n}$ für den $n$-ten Teilbruch der Kettenbruchentwicklung von $\sqrt{N}$.
	\[ \alpha_0 = \sqrt{N},\quad \alpha_{n+1} = \frac{1}{\{\alpha_n\}} \]
	Schreibe
	\[ \alpha_n = \frac{P_n + \sqrt{N}}{Q_n} \]
	mit $P_n,Q_n \in \Z$.
\end{notat*}

\begin{thm}\autolabel
	In der selben Notation wie oben gilt für $n \geq 0$
	\[ p_n^2-Nq_n^2 = (-1)^{n+1} Q_{n+1}. \]
\end{thm}

\begin{thm}\autolabel
	Sei $N \in \N,\ N \neq \square$, und $\sqrt{N} = \langle a_n,a_1,\dotsc,a_n,\dotsc\rangle$ mit Teilbrüchen $\frac{p_n}{q_n}$. Seien $x,y \in \N$. Dann ist $(x,y)$ genau dann eine Lösung der Gleichung
	\[ x^2-Ny^2=\pm 1 \]
	wenn es $n \geq 0$ gibt mit $a_{n+1} = 2 \lfloor \sqrt{N} \rfloor$ und $x=p_n,\ y=q_n$. In diesem Fall ist
	\[ x^2-Ny^2 = (-1)^{n+1}. \]
\end{thm}