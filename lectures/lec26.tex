\chapter{Primzahlen}\lecture{09.07.2021}

Wir wissen bereits, dass es unendlich viele Primzahlen gibt. Doch wie schnell wächst die Reihe der Primzahlen an?

\begin{idee*}
	Zähle Primzahlen $\leq x$ und studiere das Verhalten für $x \to \infty$.
\end{idee*}

\begin{defn*}
	Wir definieren für $x \in \R_{> 1}$
	\[ \prod(x) := |\{p \ \text{prim},\ p \leq x\}|. \]
\end{defn*}

\begin{conj*}[Gauß]
	\[ \prod(x) \sim (i \mid x) = \int_{0}^{x} \frac{dt}{\log t} \sim \frac{x}{\log x} \]
\end{conj*}

\begin{thm}\autolabel
	Sei $x \geq 2$. Dann ist \[ \sum_{p \leq x} > \log \log x - \frac{1}{2}. \]
\end{thm}

\begin{thm}\autolabel
	Sei $x \geq 2$. Dann gilt \[ \sum_{p \leq x} \frac{\log p}{p} = \log x + \Ocal(1). \]
	Sei außerdem $\theta(x) := \sum_{p \leq x} \log p$. Dann gibt es Konstanten $c_1 > c_2 > 0$, sodass für $x > 2$ gilt
	\[ c_2 x < \theta(x) < c_1 x. \]
\end{thm}

\begin{cor}\autolabel
	Es gibt Konstanten $c_3 > c_4 > 0$, sodass für $x > 2$ gilt
	\[ c_4 \frac{x}{\log x} < \prod(x) < c_3 \frac{x}{\log x}. \]
\end{cor}

\begin{lem}[Stirlingsche Formel]\autolabel
	\[ \log n! = n \log n - n + \Ocal(\log n) \]
\end{lem}

\begin{lem}\autolabel
	Schreibe
	\[ n! = \prod_{p \leq n} p^{e(p)}. \] Dann gilt
	\[ e(p) = \sum_{k \geq 1} \left\lfloor \frac{n}{p^k} \right\rfloor. \]
\end{lem}